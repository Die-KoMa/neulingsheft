\chapter{Das ABC der KoMa}

Über die Jahre hinweg haben sich einige Begriffe eingebürgert, die man auf jeder KoMa wiederfinden kann. Damit Du auch verstehen kannst, wenn die Orga von der „Kasse des Vertrauens neben dem Ewigen Frühstück“ spricht, gibt es hier eine kleine Einführung in das KoMa"=Vokabular:

\begin{description}
\item[Abschlussplenum]
    Am letzten Abend der KoMa findet das Abschlussplenum statt. Dort stellen die AKs ihre Ergebnisse vor, inhaltliche Diskussion soll möglichst nicht mehr stattfinden. Beschlüsse, z.\,B.\@ zu Resolutionen werden hier gefasst.  Freiwillige für die Ausrichtung der nächsten Konferenzen werden spätestens hier rekrutiert/bestimmt. Ein solches Abschlussplenum kann auch schon mal 4 Stunden dauern. Es gibt auf der KoMa ein Alkoholverbot im Plenum. Außerdem gibt es ein Laptop-Discouragement.

\item[Adressliste]
    In der Regel werden auf der KoMa zwei Adresslisten erstellt bzw.\ gepflegt. Eine enthält die Adressen der Teilnehmenden, meist inklusive E-Mail, sortiert nach Vorname oder Ort.  Als zweites gibt es eine Liste der Fachschaftsadressen, diese ist aber nicht öffentlich. Die Adressliste der Teilnehmenden wird üblicherweise im Rahmen des AK~Networking erstellt.

\item[Arbeitskreis]
    auch „AK“. Die meiste inhaltliche Arbeit auf der KoMa findet in den Arbeitskreisen statt. Diese werden nicht von der ausrichtenden Fachschaft organisiert, sondern von den Teilnehmenden vorgeschlagen und zum Teil auch vorbereitet. Wenn möglich sollten Arbeitskreise vor der Konferenz über die \emph{KoMapedia} angekündigt werden. Zudem sind spontane Vorschläge im Anfangsplenum sind immer möglich. In den Arbeitskreisen wird das Thema je nach Interesse und vorhandenem Material bearbeitet. Die Gruppen bestehen meist aus~5 bis 20~Personen. Die AKs laufen während der gesamten Konferenz zu Zeitslots, die die Orga am Anfang festlegt und bekannt gibt.
	
	Die Arbeitsweisen gehen	von Diskussionen über Literaturarbeit bis zu Basteln und künstlerischen Aufführungen. Dabei unterscheiden wir drei Arten von Arbeitskreisen, nämlich Resolutions AKs, Austausch AKs und Input AKs, zu denen Sachverständige (entweder die AK-Leitenden oder (evtl. externe) Referierende) die Teilnehmenden informieren, um so weitere Diskussionen oder Sacharbeit zu ermöglichen.

	Die AKs präsentieren sich auf dem Zwischen- bzw.\ Abschlussplenum und mit einem Artikel im KoMa-Kurier. Eine Ansprechperson und mögliche Orte/ Termine für Zwischentreffen (siehe WAchKoMa) werden ebenfalls auf dem Abschlussplenum bekanntgegeben.

	Typische AK-Themen die häufiger vorkamen/-kommen sind beispielsweise:
	\begin{itemize}
		\item Bachelor/Master/Diplom
		\item Studiengebühren/-beiträge
		\item Prüfungsordnung
		\item Lehramt
		\item Studentische Veranstaltungen
		\item Nachwuchswerbung
		\item Fachschaftszeitungen, -comics, -homepages
		\item AK Pella -- Dichtung und Gesang
	\end{itemize}
	Wie der Name sagt, sind Arbeitskreise vorrangig produktiv. Natürlich soll auch der Spaß nicht zu kurz kommen, allerdings sollten die „Spaß-AKs“ nicht im Vordergrund stehen, weshalb sie in der Regel keinen festen Zeislot im Tagungsablauf bekommen, sondern immer dann stattfinden, wenn genug Interessierte sich zusammen finden.

\item[Alkoholverbot im Plenum]
    Da die Dauer der Plena stark negativ mit Disziplin, Konsensfähigkeit und Konzentration korreliert, gibt es seit der 58.~KoMa in Oldenburg ein Alkoholverbot für alle Plena.

\item[Anfangsplenum]
    Mit dem Anfangsplenum beginnt offiziell die Konferenz.

    Dort gibt die ausrichtende Fachschaft zunächst organisatorische Hinweise. Dann wird von jeder vertretenen Hochschule kurz berichtet, was in der jeweiligen Fachschaft bzw. Hochschule gerade läuft, wie die hochschulpolitische Lage im jeweiligen Bundesland ist und wer von dort auf der Konferenz ist. Diese Berichte werden allerdings auch manchmal in das Zwischenplenum ausgelagert. Dann werden Vorschläge für Arbeitskreise (AKs) gesammelt und abgefragt, wie viel Interesse jeweils daran besteht. Letztlich wird festgestellt, welche AKs überhaupt stattfinden (d.\,h.\ genügend Interesse gefunden haben). Diesen werden dann Zeiten und Räume zugeteilt. Eine verbindliche Anmeldung zu den AKs erfolgt nicht.

\item[Anmeldung]
    In der Einladung werden die Teilnehmer aufgefordert, sich bei der ausrichtenden Fachschaft anzumelden, am einfachsten auf der Konferenzwebseite \url{https://die-koma.org}. Die Anmeldung sollte zeitig genug erfolgen, so dass die ausrichtende Fachschaft noch genügend T-Shirts bestellen und das Essen besser planen kann.

	Wer auf der Konferenz eintrifft, meldet sich bei der lokalen Anmeldestelle oder im Orgabüro. Diese Meldung besteht normalerweise aus: freudiger Begrüßung, Teilnahmebeitrag bezahlen, gegebenenfalls eine Quittung erhalten, Adressenliste überprüfen, Namensschild herstellen oder anstecken, eventuell Tagungsticket erhalten/kaufen, eventuell Programmheft/""Kulturheft/""Stadtplan mitnehmen.

\item[Ausrichtende Fachschaft]
    Eine Fachschaft übernimmt die Planung und Organisation der Konferenz. Dazu gehört jedoch nicht die inhaltliche Vorbereitung. Üblicherweise wird auf jeder Konferenz schon die ausrichtende Fachschaft für die übernächste Konferenz (quasi „in einem Jahr“) bestimmt.

\item[Beschlüsse]
    Beschlüsse der KoMa werden vom Plenum gefasst und sind Beschlüsse der anwesenden Personen. Sie erheben weder den Anspruch, alle Fachschaften (oder alle auf der Konferenz vertretenen Fachschaften) zu 	repräsentieren, noch für alle folgenden Konferenzen verbindlich zu sein. Letzteres ergibt sich daraus, dass die nächste Konferenz sich ja aus anderen Personen zusammensetzt. Trotzdem gibt es Beschlüsse, die die Organisation der Konferenzen betreffen und die zumindest als dringende Empfehlung an die ausrichtende Fachschaft zu verstehen sind. Schließlich sind viele, die den Beschluss mitgetragen haben, beim nächsten Mal wieder dabei. Beschlüsse werden nach dem Konsensprinzip gefasst (siehe „Konsens“).

\item[BMBF]
    Das \enquote{Bundesministeriums für Bildung und Forschung} fördert hochschulbezogene Maßnahmen, insbesondere die KoMa, sofern sie in Deutschland stattfinden. Da wir diese Förderung wenn möglich gerne warnehmen, ergeben sich für uns einige Richtlinien, an die wir uns halten müssen. Dazu gehören zum Beispiel die Erstellung eines Tagungsbandes und das genaue Führen von Teilnahmelisten (auch \enquote{BMBF-Listen} genannt) auf denen täglich alle Teilnehmenden unterschreiben müssen. Genaueres erklärt die Orga am Anfang der Konferenz. Eine Fachschaft, die an der Organisation einer KoMa interessiert ist, sollte zu diesem Thema unbedingt den Förderverein kontaktieren, da dieser für die Kommunikation mit dem BMBF zuständig ist.

\item[BuFaKo, BuFaTa] Abkürzung für Bundesfachschaftenkonferenz/-tagung. Die KoMa ist die BuFaKo der Mathematik.

\item[Einladung]
    Längere Zeit vor den Konferenzen verschickt die ausrichtende Fachschaft Einladungen über Mailinglisten und per Post an alle Fachschaften, deren Adressen bekannt sind. Darin wird vor allem der Termin bekanntgegeben, aufgefordert sich anzumelden und AKs vorzuschlagen. Etwas dichter vor den Konferenzen gibt’s dann noch eine Erinnerung via E-Mail. In der Einladung sind vor allem die Wegbeschreibung und der genaue Anfangszeitpunkt enthalten, ein Hinweis auf die Höhe des Teilnahmebeitrags sowie weitere organisatorische Details.

\item[Essen/„Ewiges Frühstück“]
    Von der Orga bereitgestellt und im Teilnehmerbeitrag enthalten ist das „Ewige Frühstück“. Dieses besteht aus einem Buffet mit Brot/Semmeln, Margarine/Butter, Marmelade, Käse, Müsli, Milch, Obst, Gemüse, etc. Dort bedienen sich alle selbst. Es steht den ganzen Tag über zur Verfügung. Darüber hinaus gibt es meist eine warme Mahlzeit am Tag -- ein vegetarisches Essen ist immer dabei. Freitags geht man meistens zusammen in der Mensa essen. Auf Sommerkonferenzen wird oft gegrillt.

	Oft stellt das Orga-Team neben dem „Ewigen Frühstück“ auch Snacks und Kaltgetränke zur Verfügung. Diese werden mit über die Kasse des Vertrauens abgerechnet und am Ende der Konferenz bei der Orga bezahlt (siehe auch „Kasse des Vertrauens“).

\item[Förderverein der KoMa e.\,V.]
    Der „Förderverein der KoMa~e.\,V.“ wurde auf der 63. KoMa in Paderborn gegründet. Er ist gemeinnützig und sein Ziel ist es, die Organisation der KoMa zu unterstützen. Als solcher kümmert er sich um Spenden, Anträge auf Bundesfördermittel oder auch um Sponsoren. Jeder, der die KoMa unterstützen will, kann gerne dem Verein beitreten, oder spenden. Die Vereinssitzungen finden üblicherweise während der KoMa statt.

\item[Geschäftsordnung]
    Die KoMa hat keine Geschäftsordnung oder Satzung. Verfahrensweise und Struktur können sich auf jeder Konferenz daher ändern.

\item[Getränke]
    Kaffee, Tee, Milch und Wasser gehören zum Frühstück und müssen nicht extra bezahlt werden. Weiter gibt es Bier, Saft und gelegentlich Wein. Diese werden, wie die Schokoriegel, über eine Strichliste abgerechnet und am Ende der Konferenz bei der Orga bezahlt (siehe auch „Kasse des Vertrauens“).

\item[Handzeichen]
    Zur Verbesserung des Diskussionsablaufs wurden Handzeichen vereinbart, die z.\,B. Zustimmung oder Ablehnung signalisieren, ohne Krach	zu machen. Details dazu gibt es auf \autopageref{sec:handzeichen}.

\item[Infoheft]
    Ein Heft mit langer Tradition: Bei der Anmeldung erhalten die Teilnehmer das Infoheft. Darin sind organisatorische Hinweise aufgeführt, das Programm, Wegweiser und Tipps für das Abendprogramm. Sofern es solche gibt, können auch ein Stadtplan, Kulturübersicht,	Geschichtsabriss etc.\ zum Infoheft gehören.

\item[KaWuM]
    KaWuM, [die], Konferenz aller werkstofftechnischen und materialwissenschaftlichen Studiengänge

\item[Kasse des Vertrauens]
    auch „KdV“. Neben den Getränken hängt eine große Liste, in die sich alle eintragen und für ihre Getränke und Snacks Striche machen.  Bezahlt wird vor der Abreise bei der Orga.  Wasser ist traditionell kostenlos und wird daher nicht in die Strichliste eingetragen.	

\item[KIF]
    KIF, [die], Konferenz der (deutschsprachigen) Informatikfachschaften.

	Die Entsprechung der KoMa für Informatiker. Die Nummerierung der Konferenzen besteht aus den Jahreszahlen seit der ersten KIF. Teilnehmende der KIF werden nicht als „Kiffer“, sondern als „KIFfel(s)“ bezeichnet. KIF und KoMa haben eine lange freundschaftiche Verbundenheit und finden manchmal sogar zeitglich am gleichen Ort statt.

\item[Kneipentour]
    Häufig gibt es eine Kneipentour, die von der ausrichtenden Fachschaft organisiert wird. Bei einem schönen Abend können die Teilnehmenden einander und die Stadt in der die KoMa stattfindet kennenlernen. Am nächsten Tag geht es aber wieder früh weiter mit den AKs, denn wer trinken kann kann auch arbeiten!

\item[KoMa]
    Koma, 1 [die], um den Kern eines Kometen liegende Nebelhülle (Gasatmosphäre). -- 2 [die], Bildfehler bei Linsen oder Linsensystemen: Seitlich der optischen Achse gelegene Punkte werden nicht punktförmig sondern in Form eines Kometenschweifes abgebildet. -- 3 [das], Koma, tiefe Bewusstlosigkeit, z.\,B. bei Zuckerkrankheit, Harnvergiftung, u.\,a.\ (Quelle: Bertelsmann Universallexikon via „Koma für Neulinge“ [Dank an Gesa aus Köln])

	Und die wichtigste Bedeutung: -- 4 [die], Konferenz der deutschsprachigen Mathematik"=Fachschaften.

	Letztere bezeichnet in erster Linie die Zusammenkunft der Teilnehmende einmal pro Semester. Es gibt eine Sommer-KoMa und eine Winter-KoMa. Über den Zusatz „deutschsprachigen“ wurde auf der KoMa in Bonn (WS94/95) mal diskutiert, mit dem Ziel, nicht nationalistisch zu sein. Auf ausdrücklichen Wunsch der Teilnehmenden aus Österreich wurde er dann aber beibehalten, weil nur so klar wird, dass die KoMa keine reine Bundesfachschaftentagung ist. Schließlich kommen regelmäßig Personen aus Österreich und der Schweiz zur Konferenz. Die erste KoMa war im WS 1977/78\footnote{Der Ort konnte bisher nicht festgestellt werden, ein Protokoll existiert jedoch.} noch unter dem Namen \emph{VDS"=Fachtagung Mathematik}. In den 80ern änderte sich dieser Name in \emph{Bundesfachschaftentagung Mathematik}, bevor der heutige Name entstand.  Bis 2005 wurde zur Nummerierung die Zählung nach Paulus (n.\,P.) verwendet, dem Rekordbesucher der KoMa, welcher alle von ihm besuchten KoMata als Grundlagen einer Nummerierung wählte. Nach Recherchen im KoMa-Archiv konnte aber die exakte Anzahl der stattgefundenen KoMata bestimmt werden\footnote{Die Zählung nach Paulus musste dadurch um~6 nach oben korrigiert werden.}. Die korrigierte Anzahl wurde als neue Nummerierung übernommen. Als Pluralbildungen sind KoMata, KoMen, und, seltener, KoMae in Gebrauch. Auf der KoMa~63 wurde sich aber auf den Begriff „KoMata“ als Pluralform geeinigt, um dieses Chaos zu beenden.

\item[KoMa-Archiv]
    Vieles hat sich über die letzten Jahrzehnte angesammelt. Alle Daten, Publikationen und Akten der KoMa werden daher im KoMa-Archiv aufgehoben und gepflegt. Das Archiv wird vom KoMa-Büro verwaltet und gepflegt.

\item[KoMa-Büro]
    Eine Fachschaft verwaltet die an die KoMa gerichtete Post und verschickt den KoMa-Kurier, sofern dieser nicht mit den Einladungen verschickt wird. Das KoMa-Büro befindet sich zur Zeit an der Uni Bonn. Es dient auch als Geschäftsadresse des Fördervereins der KoMa~e.\,V.

\item[KoMa-Kurier]
    Der KoMa-Kurier (früher auch KoMa-Kuhrier geschrieben) ist eine Art Zeitung, die an möglichst alle Fachschaften verschickt wird. Er besteht vor allem aus Protokollen und AK"=Berichten der jeweils letzten KoMa, dem legendären Vorwort und allem, was sonst noch Leute so beisteuern.

\item[KoMapedia]
    Die KoMapedia ist das Wiki der KoMa und wird zur Dokumentation von Arbeitskreisen und KoMata genutzt. Sie ist über die KoMa"=Webseite	die-koma.org erreichbar.

\item[Konsens]
    Konsens heißt nicht, dass alle einer Meinung sind. Konsens heißt, eine Entscheidung zu treffen, mit der alle leben können. Kein Konsens liegt vor, wenn eine Person ein Veto einlegt. In diesem Fall ist kein Beschluss gefasst.  Es ist aber z.\,B. möglich, dass diejenigen, die etwa eine Resolution befürworten, diese jetzt privat unterschreiben und veröffentlichen, aber eben nicht als KoMa.

\item[Laptop-Discouragement]
    Auf der 77. KoMa wurde ein Laptop-Discouragement beschlossen, also die Bitte,     nur dann einen Laptop vor euch stehen zu haben, wenn ihr ihn für produktive Plenumszwecke (z.B. zum       Protokollieren) braucht, um zu vermeiden andere Teilnehmende abzulenken.

\item[Mail-Verteiler]
    Die Mailadresse um die aktiven KoMatiker zu erreichen ist \mail{aktive@die-koma.org}, in den ihr euch auf jeden Fall eintragen solltet um auf dem Laufenden zu bleiben. Weitere Informationen dazu findet ihr direkt über die KoMa-Webseite.

\item[Meinungsbild]
    Im Plenum wird manchmal gefragt, „Wer ist dafür/wer ist dagegen?“, um festzustellen, ob überhaupt Bedarf oder die Möglichkeit besteht, eine bestimmte Entscheidung zu treffen. Dies ist kein Beschluss! Das Meinungsbild soll lediglich allen die Möglichkeit geben, zu sehen, wie die anderen gerade denken. Da es das Konsensverfahren durcheinander bringen kann, weil es wie eine Abstimmung aussieht, wird es auch kritisch gesehen. Dies passiert in der Regel mit den gebräuchlichen Handzeichen.

\item[Mörderspiel]
    Das Mörderspiel ist ein einfaches Kennenlernspiel, dass mit viele Personen gespielt werden kann. Jeder hat den Auftrag, eine bestimmte Person zu \enquote{ermorden}, also einen Gegenstand direkt zu übergeben.
    Das Spiel fördert dient als Anlass sich kennenzulernen und erzeugt nebenbei einige seltsame Situationen.
    Die genauen Regeln werden meist im Tagungsheft spezifiziert.
    
\item[Namensschild]
    Bei der Anmeldung bekommen alle ein	Namensschild. Darauf steht der Vorname und die Hochschule, manchmal auch ein Spitzname und z.B. der jeweilige Twittername. Manchmal finden sich im Namensschild auch Gutscheine oder andere nützliche Hinweise. Das Namensschild wird zwecks besserer Kontaktaufnahme während der ganzen Konferenz getragen. Außerdem ist es meist für Mitarbeitende der örtlichen Hochschule ein relevantes Werkzeug um die Konferenzteilnehmenden als solche zu identifizieren.

\item[Orga]
    Wahnsinnige, die einen Moment lang nicht oder zu wenig nachgedacht haben. Diejenigen, die die Konferenzen vorbereitet haben und für die Organisation zuständig sind. Oft durch spezielle Namensschilder oder T-Shirts gekennzeichnet.
    Da die Orga für alles, was die Teilnehmenden machen verantwortlich ist, hat sie stets das letzte Wort, wenn es um Verhaltensregeln vor Ort geht.

\item[Plenum]
    Im Plenum treffen sich alle Teilnehmenden, um gemeinsam Informationen auszutauschen und zu diskutieren. Vom Plenum werden Beschlüsse gefasst. Es gibt ein Anfangs-, ein Zwischen- und ein Abschlussplenum. Die Teilnahme am Plenum ist natürlich, wie alles andere auch, freiwillig, trotzdem ist es stark erwünscht, dass alle daran teilnehmen.
    
    Plena sind wichtig, da auf ihnen ausgetauscht wird, was bisher auf der KoMa gemacht wurde und wie weiter vorgegangen werden soll. Außerdem werden sämtliche Beschlüsse der KoMa in den Plena gefasst.
    
	Protokoll und Moderation übernehmen in der Regel einzelne, erfahrene, Teilnehmende.
    
    In den Plena der KoMa herrscht Alkoholverbot und Laptop-Discouragement!

\item[Protokoll]
    Dokumentiert Geschehenes sprachlich neutral, objektiv und allumfassend. Üblicherweise wird das Protokoll durch Freiwillige in einem Etherpad erstellt. Am Protokoll mitzuschreiben ist der einzige allgemein akzeptierte Grund, im Plenum einen Laptop vor sich stehen zu haben.
	Zu Beginn der Konferenz sollte eine einheitliche Nomenklatur für die URL von AK-Protokollen ausgemacht werden, damit alle Teilnehmenden die Möglichkeit haben, die Protokolle zu finden und zu lesen.

\item[Resolution]
    Eine gemeinsame Stellungnahme der KoMa (d.\,h.\ der dort anwesenden Menschen) zu meist (hochschul-)politischen Themen, wird auf dem Abschlussplenum beschlossen.
    Diese wird veröffentlicht (Presse) und an jeweilige Ministerien/Regierung usw. verschickt.
    Resolutionen werden vor Beginn des Abschlussplenums ausgehängt, damit alle sie lesen können. Traditionell gibt es fast immer mindestens eine Resolution auf der KoMa.

\item[Schlafquartiere]
    Zum Schlafen bringen die Teilnehmenden Schlafsack und Isomatte mit. Wenn möglich gibt es ein gemeinsames Schlafquartier in geeigneten Räumen, z.\,B. Turnhalle oder Jugendzentrum. Wenn es nicht anders geht, wurden die Teilnehmenden auch schon mal einzeln oder in kleinen Gruppen bei einheimischen Studis oder WGs untergebracht. 
    Im Allgemeinen sind die Teilnehmenden nicht sehr anspruchsvoll, Nähe zwischen Frühstücks-/Tagungsraum und gemeinsamer Unterkunft wird jedoch bevorzugt.

\item[Spenden]
    Falls ihr Interesse habt dem Förderverein etwas zu spenden, könnt euch etwa folgendes         überlegen: Wenn ihr auf eine KoMa fahrt und	aus einer der Hochschulen kommt die eure Auslagen erstatten, dann könnt ihr das Geld, das ihr daheim sowieso für Essen ausgegeben hättet, dem Förderverein spenden und damit die Ausrichtung der kommenden KoMata und die Finanzierung von WAch-KoMaTa zu unterstützen.

\item[Spiel]
    Einziges Ziel ist, das Spiel zu vergessen. Wer sich daran erinnert, muss umstehende Personen darauf aufmerksam machen, \enquote{das Spiel verloren} zu haben. Wer das liest, hat das Spiel verloren.

\item[Stadtführung]
    Die ausrichtende Fachschaft veranstaltet meist eine Stadtführung. Sie wird in der Regel von einheimischen Studis geleitet. Dabei liegt der Schwerpunkt nicht unbedingt auf touristischen Attraktionen, sondern auf einem Einblick in den Hochschulort und das zugehörige Studileben.
    
    Manchmal werden auch einzelne Exkursionen zu bestimmten Themen als Alternative zur Stadtführung angeboten.

\item[Studentischer Akkreditierungspool]\label{itm:pool}
    In Deutschland müssen alle Bachelor-und Masterstudiengänge akkreditiert werden. An solch einer Akkreditierung sind auch immer Studierende beteiligt, welche vom unabhängigen „Studentischen Akkreditierungspool“ (\url{https://www.studentischer-pool.de}) zugeteilt werden. Die KoMa entsendet in ihrer Rolle als Bundesfachschaftentagung Mathematikstudierende in diesen Pool.

\item[Tagungsticket]
    (oder auch „Konferenzticket“) Je nach Möglichkeit und Notwendigkeit (Verkehrsangebot, Lage von Schlaf- und Tagungsräumen, Preis) gibt es zu den Konferenzen ein Tagungsticket für den öffentlichen Naherkehr. Dieses muss eventuell zusätzlich zum Teilnehmerbeitrag bezahlt werden.

\item[Teilnehmerbeitrag]
    Zur Finanzierung der Konferenzen (Essen, Unterkunft, etc.) zahlen die Teilnehmenden einen Beitrag. Dieser lag in den letzten Jahren immer zwischen 25\,€ und 30\,€. Ein Tagungsticket muss eventuell extra bezahlt werden. Um das Geld ggf. vom AStA/StuRa/Konvent oder der Hochschule erstattet zu bekommen, gibt es eine Quittung bzw. Teilnahmebescheinigung.

\item[Teilnehmende]
    Menschen, die an den Konferenzen teilnehmen. Zur Teilnahme ist es weder Pflicht, einen mathematischen Studiengang zu studieren, noch bei irgendeiner Fachschaft aktiv zu sein.

\item[Termin]
    Die Konferenzen gehen in der Regel von Mittwoch Abend bis Sonntag Vormittag. Die Sommer"=Konferenzen finden meist über einen freien Donnerstag Ende Mai/Anfang Juni statt, die Winterkonferenzen etwa Anfang/Mitte November.

\item[T-Shirts]
    Es werden vom Orga-Team T-Shirts für die Konferenzen bedruckt. Ein T-Shirt ist üblicherweise im Teilnehmahmebeitrag enthalten, manchmal kann man bei der Anmeldung angeben, dass man weitere T-Shirts möchte.

\item[Veto]
    Wer bei einer Konsensentscheidung mit einem Beschluss überhaupt nicht einverstanden ist kann ein Veto einlegen. Mit einem Veto ist kein Konsensbeschluss möglich.

\item[WAchKoMa]
    Moderner Name für „Zwischentreffen“. Es bedeutet „Weiterführung	von Arbeitskreisen unter chaotischen Verhältnissen der Konferenz der deutschsprachigen Mathematikfachschaften“. 
    
    Einige AKs treffen sich auch zwischen zwei Konferenzen noch mal. Das Treffen wird von den AK"=Mitgliedern selbst organisiert und ist in der Regel auch offen für Personen, die auf der Konferenzen nicht in dem AK waren.  Eine grobe Planung für Ort und Termin wird meist schon auf dem Abschlussplenum bekanntgegeben, genaueres gibt es üblicherweise über die Mailingliste(n).

\item[ZaPF]
    ZaPF, [die], Zusammenkunft aller Physik-Fachschaften

\item[Zwischenplenum]
    Auf der KoMa gibt es zusätzlich zu Anfangs und Abschlussplenum ein Zwischenplenum am Freitagabend.  Dort gibt es Berichte und/oder Diskussionen zu den AKs, die schon stattgefunden haben und Resolutionen. Wenn Fachschaften zu spät anreisen, haben sie hier noch mal die Möglichkeit, sich vorzustellen.
\end{description}
