\chapter{Was ist die KoMa?}

Wenn man studiert, dann lernt man meistens nur die eigene Uni oder FH kennen,
einen beschränkten Kreis von Professorinnen, Professoren und Mitstudierenden.
Und man ist gefangen im Rhythmus von Fachvorlesungen, Seminaren und
Übungszetteln.  Das Gleiche trifft leider auch oft für die Mitglieder in der
studentischen Selbstverwaltung zu. Selten ergibt sich die Gelegenheit über die
lokale Hochschulpolitik hinauszuschauen und zu sehen was an deren Unis besser
oder schlechter läuft. Und könnten gerade solche Einblicke neue Impulse und
Verbesserungen an der eigenen Hochschule bezwecken.

Um diesem Problem entgegenzutreten gibt es die KoMa. Denn es ist ein
essentieller Erfolgsfaktor, ob man einen größeren Blick über die eigene
Hochschule hinaus besitzt oder nicht. Denn betrachtet man das Studium, so ist
doch auch Teil eines Hochschulbetriebes. Und die Qualität unseres Studiums
hängt von den Rahmenbedingungen ab, die sowohl an unseren Hochschulen, als auch
auf Bundes- oder Landesebene geschaffen werden.  Andere Hochschulen können
etwas besser machen oder andere können von uns lernen. Das gilt im Kleinen bei
Fragen wie der Orientierungswoche oder der Fachschaftszeitschrift, hört aber
lange noch nicht bei den großen Fragen wie der Ausgestaltung von
Studiengangsakkreditierungen auf.

Doch auch neben diesen hochschulpolitischen Themen gibt es noch einen ganz
gewichtigen weiteren Grund sich mit anderen Mathematikstudierenden zu treffen
und auszutauschen: Mathematik lebt vom Austausch und von einer Diskussion über
die Themen, das Lernen von Mathematik und das „herumspinnen“ mit mathematischen
Fragestellungen. Wie könnte das besser gehen, als wenn man fünf Tage lang einen
Haufen Mathematikstudierende zu einer Konferenz versammelt?

\section{Ziele der KoMa}

Um einmal über den Tellerrand des eigenen Studiums hinaus zu blicken und ein
paar andere Leute zu treffen, die dasselbe studieren oder sich auch für
Mathematik interessieren, gibt es einmal im Semester die „Konferenz der
deutschsprachigen Mathematikfachschaften“ (KoMa). Das ist ein sehr förmliches
Wort für eine lockere Sache. Studis und Interessierte treffen sich einfach für
ein paar Tage, diskutieren über Aspekte des Fachs, die sonst so im Studium
nicht vorkommen, über Uni- und andere Politik und über alles, wozu wir gerade
Lust haben.

Zur kreativen Spannung auf den Konferenzen gehört es auch, dass erst vor Ort
wirklich rauskommt, welche Themen von Interesse sind. Auch völlig neue,
spontane Arbeitskreise bilden sich gelegentlich. Daneben haben wir natürlich
auch eine Menge Spaß, lange Abende in den Kneipen, bei Spielen, beim
gemeinsamen Grillen oder bei netten Unterhaltungen mit Gleichgesinnten.

Natürlich ist das ganze Programm freiwillig. Jeder und jede macht worauf er/sie
gerade Lust hat. Schließlich sollen die Konferenzen nicht nur interessant sein,
sondern auch Spaß machen.  Daneben erfährt man, wenn man Leute aus ganz
Deutschland, Österreich und (manchmal auch) der Schweiz trifft, viel darüber,
wie anderswo ein Mathematikstudium aussieht oder was sonst so los ist in der
weiten Welt.
